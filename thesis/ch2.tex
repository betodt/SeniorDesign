\chapter{Design}

\section{Conceptual Models}
In this section, we discuss how some of the main components and processes of our system will function and behave. These models will lay the foundation for the initial development of our system. We included three models to clearly illustrate our vision for the system; an entity relationship diagram, activity diagrams, and a mock-up of our primary application view.
\subsection{Entity Relationship Diagram}
An entity relationship diagram describes the types of connections between the main components of the system. Figure \ref{erd} shows us the five primary pieces that makeup our application; projects, teams, team members, a team leader, and project tasks. A project and a team can only have one leader. A project can have many teams, which are composed of many members. Lastly, members can have many tasks. 
A team leader effectively makes the projects. He or she is the person who initially uses the application to enter all the details of the project, including tasks and deadlines. The team leader also invites other members to the application and starts to assign them tasks and even add to them to a team with its own set of subtasks. 
\begin{figure}[ht]
\centering
\empuse{erdiag}
\caption{Entity Relationship Diagram}
\label{erd}
\end{figure}
\FloatBarrier

\subsection{Activity Diagrams}
An activity diagram is a flowchart of a user\textsc{\char13}s actions to accomplish a goal. Our first figure, figure \ref{activityOne}, depicts the actions necessary to create a project. Assuming that the user has signed in, the user must name her project, and also define the project\textsc{\char13}s goal and deadline. Next she will have the option to add team members to the project. (This process will be explained in Figure \ref{activityTwo}).  If she declines to add members, the project creation process is completed. 

\begin{figure}[ht]
\centering
\empuse{activityR1}
\caption{Activity Diagram: Creating a project.}
\label{activityOne}
\end{figure}
\FloatBarrier
\par The next activity diagram, figure \ref{activityTwo}, describes team creation. If the user wishes to add team members of his project, he must first add a team description. Next, he/she will add a team member by entering his/her email. If the member has an account in our application, he/she will be successfully added to the team. If he/she is not a member, our system will send him/her an e-mail invitation. The user will then have the option to continue adding team members. Lastly, the project creator will assign a project lead role to either him/herself or another member. Once the project lead role is assigned, he/she has completed the team creation process. 


\begin{figure}[ht]
\centering
\empuse{activityR2}
\caption{Activity Diagram: Creating a team.}
\label{activityTwo}
\end{figure}
\FloatBarrier

\par The next activity diagram, figure \ref{activityFour}, depicts the assignment of  tasks process. The user will select a task and proceed to assign the task to team member(s). After this step, the process is complete. 

\begin{figure}[ht]
\centering
\empuse{activityR4}
\caption{Activity Diagram: Assigning tasks.}
\label{activityFour}
\end{figure}
\FloatBarrier


\subsection{Mockups}

Our final model, the mock-up, gives us a rough overview of the layout for the application. The nav bar and left side bar are minimalistic, having only as many features as necessary to accomplish any logistics such as account management and login buttons. The main feature, a blank canvas, will provide users with any interactivity related to the current projects they are viewing, including tasks, timelines, and upcoming deadlines. 

\begin{figure}[ht]
\includegraphics[width=\textwidth]{mockup.png}
\caption{A mockup of our system concept.}
\end{figure}
\FloatBarrier

\section{Use Cases}
A use case defines the steps required to accomplish a specific goal. The following use cases describe how the user interacts with the system to achieve these goals, including preconditions, postconditions, steps required, and common errors that might occur. The use case diagram, Figure \ref{ucd}, helps to illustrate the user\textsc{\char13}s major actions in our application: creating a project, adding team members, and creating and assigning tasks. 
\begin{figure}[ht]
\centering
\empuse{usecasediag}
\caption{Use Case Diagram}
\label{ucd}
\end{figure}

\begin{figure}[ht]
\begin{usecase}

\addtitle{Use Case 1}{Create Project} 

\addfield{Goal:}{Initialize project objectives, tasks, and deadlines}

\addfield{Actor:}{User}

\addfield{Preconditions:}{Account is set up and logged in}
%when multiple
%\additemizedfield{Preconditions:}{} 

\addfield{Postconditions:}{Project is initialized}
%when multiple
%\additemizedfield{Preconditions:}{}

%Main Success Scenario: A typical, unconditional happy path scenario of success.
\addscenario{Steps:}{
\item Name the project
\item Define project goals and deadlines
\item Add people to the project
\item Save the project
}

%Extensions: Alternate scenarios of success or failure.
\end{usecase}
\end{figure}

\begin{figure}[ht]
\begin{usecase}

\addtitle{Use Case 2}{Add team members} 

\addfield{Goal:}{Associate people with the project}

\addfield{Actor:}{User}

\addfield{Preconditions:}{Logged in, project created, members added}
%when multiple
%\additemizedfield{Preconditions:}{} 

\addfield{Postconditions:}{Project has people assigned}
%when multiple
%\additemizedfield{Preconditions:}{}

%Main Success Scenario: A typical, unconditional happy path scenario of success.
\addscenario{Steps:}{
\item Enter team description
\item Enter member contact
\item If found, add member to team, if not found send e-mail invite
\item Once, all members are added, choose a leader
\item Submit
}

%Extensions: Alternate scenarios of success or failure.
\addscenario{Exceptions:}{
\item[A.] Project does not exist:
\begin{enumerate}
\item[1.] System shows failure message
\item[2.] User taken to project creation phase
\end{enumerate}
\item[B.] Project does not have members:
\begin{enumerate}
\item[1.] System shows failure message
\item[2.] User prompted to add members
\end{enumerate}
}
\end{usecase}
\end{figure}

\begin{figure}[ht]
\begin{usecase}

\addtitle{Use Case 1}{Assign tasks} 

\addfield{Goal:}{Assign tasks to individual members or teams}

\addfield{Actor:}{User}

\addfield{Preconditions:}{Logged in, project created, tasks created}
%when multiple
%\additemizedfield{Preconditions:}{} 

\addfield{Postconditions:}{Tasks assigned}
%when multiple
%\additemizedfield{Preconditions:}{}

%Main Success Scenario: A typical, unconditional happy path scenario of success.
\addscenario{Steps:}{
\item Select a task
\item Select members who will be assigned the task
\item Save the task
}

%Extensions: Alternate scenarios of success or failure.
\addscenario{Exceptions:}{
\item[A.] Project does not exist:
\begin{enumerate}
\item[1.] System shows failure message
\item[2.] User taken to project creation phase
\end{enumerate}
\item[B.] Project does not have tasks:
\begin{enumerate}
\item[1.] System shows failure message
\item[2.] User prompted to add tasks
\end{enumerate}
}
\end{usecase}
\end{figure}
\FloatBarrier

\section{Architectural Design}
\begin{figure}[ht]
\includegraphics[width=\textwidth]{archDiag.png}
\caption{Diagram describing the technology stack for our application.}
\end{figure}
\FloatBarrier

\section{Technologies Used}
In this section, we describe the technologies we chose to develop our application and the features they provide for us.

\begin{enumerate}
\item HTML5 \par The latest version of the standard web markup language
\item JavaScript \par A high-level, untyped, and interpreted programming  language supported by all modern web browser
\item CSS3 \par A stylesheet language used to describe the presentation of HTML documents
\item Ember.js \par A front-end javascript framework based on the model-view-controller (MVC) model and applies programming conventions to build scalable applications
\item Ember-CLI \par A command line utility that provides a fast asset pipeline.
\item  Firebase \par A backend-as-a-service that provides client-side APIs, and services such as databases, web hosting, and authentication.
\end{enumerate}

\section{Design Rationale}
In this section, we describe why we chose these technologies and the advantages and disadvantages associated with each of those technologies. 
\subsection{Technology Rationale}
\begin{enumerate}
\item HTML5/CSS3 /JS \par This trio of web technologies has become the standard for web development
\begin{enumerate}
\item Advantages
\begin{enumerate}
\item Well documented and supported in the development community
\item Gives the developer control over the look and feel of the application
\item Compatible with nearly any web browser
\end{enumerate}
\item Disadvantages
\begin{enumerate}
\item Without a framework or library it has little functionality
\item Requires a lot of time and effort to create a complete and robust application
\item Some browser interpret certain elements differently
\end{enumerate}
\end{enumerate}
\item Ember.js \par We chose a framework because it expedites the development process. Ember.js boasts taking its developers 80\% of the way through development with the use of best idioms and proper practices. It is used by companies such as Yahoo, Groupon, and Square\cite{ember}.
\begin{enumerate}
\item Advantages
\begin{enumerate}
\item Faster development start time
\item Allows the developer to create reusable components through the use Handlebars templates
\item The built in router allows developers to create multi page applications by navigating the user through states rather than physical webpages, reducing the need to keep track of these different states since the page never physically reloads. 
\end{enumerate}
\item Disadvantages
\begin{enumerate}
\item The framework is based around conventions, limiting developers freedom for a majority of the development process.
\item If the developers do stray from convention, they could find themselves having a difficult time reaching completion.
\end{enumerate}
\end{enumerate}
\item  Firebase \par We chose this backend service due to the fact that it provides a client-side API for Ember.js called EmberFire.
\begin{enumerate}
\item Advantages
\begin{enumerate}
\item Using a client-side API eliminates the need for a backend language and allow us to focus on a single language for all of our development.
\item Firebase also provides hosting and database services for free!
\end{enumerate}
\item Disadvantages
\begin{enumerate}
\item Support is outsourced, we do not have direct access to our server and are at the mercy of whatever functionality is provided by the API
\item A free firebase account does not provide provide private backups of our data, increasing the probability of data loss
\end{enumerate}
\end{enumerate}
\subsection{Aesthetics Rationale}
\begin{enumerate}
\item[1.] \textbf{Existing Product Aesthetics}
\begin{enumerate}
\item[a.] \textbf{Social Media Aesthetics} Social media attracts users from a wide variety of backgrounds. Good social media encompasses design that is usable by literally anyone around the world. Thus, we want to make our design use components of social media because our audience will generally be familiar with this layout.\cite{usability}
\item[b.] \textbf{Competitor Aesthetics} In addition, we explored a competitor\textsc{\char13}s application: Microsoft Project\cite{msproject}.
\FloatBarrier
\begin{figure}[ht]
\includegraphics[width=\textwidth]{msoftp.png}
\caption{The interface for Microsoft Project}
\end{figure}
\FloatBarrier
 
\begin{table}[ht]
\centering
\begin{tabularx}{\textwidth}{|X|X|X|}
  \hline
  \textbf{The Good} & \textbf{The Bad} & \textbf{The Ugly} \\
  \hline
  \begin{itemize}
    \item[-] Familiar images
    \item[-] Color coordination
    \item[-] Feature placement
  \end{itemize} &
  \begin{itemize}
    \item[-] Information overload
    \item[-] Cluttered workspace
    \item[-] Unnecessary features
  \end{itemize} &
  \begin{itemize}
    \item[-] Restrained within a \textquoteleft{window}\textquoteright
    \item[-] Nearly 20 year old design!
  \end{itemize} \\
  \hline
\end{tabularx}
\caption{A quick analysis of Microsoft project.}
\end{table}
\FloatBarrier
While Microsoft\textsc{\char13}s project management tool has some good design ideologies highlighted in the social media section, it generally gives the user more than he or she needs at a given time. The workspace looks crowded and at time messy, in no way representing any form of organization. To cap it all off, while graphics have improved over the years, the actual design of the product remains exactly the same.
\end{enumerate} 
\item[2.] \textbf{Our Proposed Aesthetics} 
\par We want our application to have low learning curve and to achieve this, we must make our product user-friendly. In order to achieve the desired usability, our system will have to meet specific criteria:
\begin{enumerate}
\item The design of the individual components will afford their features, i.e. hamburger-buttons will serve as an interface to a menu and text areas will serve as a canvas for writing. Familiar icons increase usability since most audiences will understand the significance of these icons.
\item The system will experience minimal latency from user input to interface response. Our application needs to be as immersive as possible. Any technical glitch will detract from the user\textsc{\char13}s experience.
\item The systems design will follow conventional web application layouts. Proper placement of features will be the difference between leaving users confused and onboarding new users so quickly that they are already planning activities in a matter of minutes.
\end{enumerate}
\end{enumerate}
\end{enumerate}