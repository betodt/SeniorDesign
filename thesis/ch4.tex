\chapter{Conclusion}

\section{The Final Application}
Planly is a collabortive project management application built as a single page web application. When users type in the URL for Planly, they are taken to the application's homepage shown in Figure 4.1. The homepage featured an image of our application. It is very minimalistic as users need to have an account in order to use the application.
\subsection{Login}
In order to log in or sign up for Planly, users must select ``Sign In'' on the top right corner of the homepage. We provide users with three different methods to sign up for an account: Google, Facebook, or Email, shown in Figure 4.2. After a user has successfully logged in, he/she will be presented with the \emph{Projects} view. The \emph{Projects} view, seen in Figure 4.3, hosts all of the user's projects as well as the members for each project. 
\subsection{Creating Projects and Teams}
From the \emph{Projects} view, users can create a new project as well as create teams for the project. By clicking on the + icon, users will activate the Project Creation form, depicted in Figure 4.4. The user needs to input the Project name, goal, deadline and team members. After completing the Project Creation form, users are given the option to create a team or to skip it at this time. 
\par If the user decides to create a team, he will need to enter the team name, description and team members. 
\subsection{Adding Tasks and Subtasks}
After the project has been created, users can now add task and subtask to the project. Figure 4.5 shows the \emph{Tasks} view. In the \emph{Tasks} view, users are presented with the project's name, progress bar, and a form to add tasks and subtasks. Once a user fills out the tasks form, he can then add a subtask or simply create the task. The task will appear on the screen with the task's deadline and who the task is assigned too. From there, users can click on the subtask icon to see the subtasks associated with the task or then can click the comment icon and post a comment. 
\par When a user marks a task as completed, the progress bar will change to reflect the progress of the project. 


\section{User Testing and Results}
After we completed Planly, we hosted a user testing session where we invited students to complete a series of tasks and provide us with their feedback. At the end of the test, we asked users to submit an anonymous survey. Figure 4.6 shows our user testing results. 
\par 
\section{Lessons Learned}
\subsection{Challenges}
\subsection{Future Work}
