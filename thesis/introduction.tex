\chapter{Introduction}

\section{Motivation}

Individuals currently live in a society that revolves around productivity. People from managers to students are bombarded with numerous tasks and daunting deadlines, making it extremely difficult to stay organized. In addition, people have multiple commitments in their lives, adding to the stress of keeping up with their already busy schedules. Currently, individuals are faced with a variety of organization methods. However, the two tools most people resort to are e-mail and notepads. Without a doubt, people are often drawn to the simplicity of typing out a quick note or sending an e-mail. However, it becomes extremely difficult to keep track of all these notes and e-mails. Before you know it, lines of communication become broken and trains of thought are lost forever. Older conventions are becoming outdated as e-mail searches often yield innumerable results, and text files are never where they need to be. Team members need a way arrange their tasks and project managers require a solution to view the process of projects and ensure that every member is on track. 

\section{Current Solutions} 

Previous solutions have failed to provide an accessible and simple project management tool. One crucial issue is that most workflow management solutions, such as Asana and Basecamp, are often marketed and catered to large enterprises rather students and small groups.These solutions are exceedingly expensive, thus community organizations and students tend to avoid these applications. Another issue with current solutions is their steep learning curves that often require additional training. In addition, present solutions, such as Pivotal Tracker and Microsoft Outlook, are not straightforward and can make it difficult to setup a meeting or look up a contact. Furthermore, present project management applications are visually unappealing and not user-friendly. Most applications follow the conventional three panel layout, as seen in Outlook, pushing your tasks into some sidebar crevice where these tasks can easily be overlooked. These applications are not designed for students or small groups with minimal project management experience.

\section{Proposed Solution}

We propose a simple, yet powerful, web-based solution that lives entirely within a browser. The goal is to have an easily accessible place for not only small groups of students or professionals to host a collaborative environment, but, we also want individuals to use our system for personal projects. Our accessibility will be achieved through a simple sign-on system without the hassle of painful payment systems and complex learning curves. Instead of struggling to learn the software, the user will be guided through a quick, interactive tutorial where he or she will establish goals, deadlines, and other project management tasks. The user will then be able to begin collaborating by creating groups and inviting other team members. Here, the user has the ability to add deadlines as well as assign internal tasks for him/herself or external tasks for others to view. The application will include a progress bar showing where the project is, where it was, and most importantly, where it needs to go. Current solutions provide users with a static page of checklists. In our application, when progress is made, the interface will reflect those changes. The timeline will expand into greater detail, time sensitive reports will become available, and completed tasks and files will be, literally, lost in time.  We will allow the user to travel through their project timeline, revealing a snapshot of previous or future tasks. Lastly, to further promote organization, we plan to allow teams to store files within their project portals, eliminating time spent searching for important documents. Our product will be tested in live working environments, showing how teams use our solution to better organize their projects in an effort to help us help them achieve their organizational goals. 

\section{Requirements}
We defined a set of functional and nonfunctional requirements for our web application. Functional requirements specifies criteria that can be used to judge the operations of a system. In other words, functional requirements define what must be done by the system. Nonfunctional requirements describe the behavior and the limits of the application. To prioritize our requirements, we conducted an Analytic Hierarchy Process (AHP) chart seen in table \ref{ahp}. By using AHP, we were able to easily identify which requirements were critical, recommended and suggested.
\FloatBarrier
\begin{table}
\begin{tabular}{c|c|c|c|c|c|c|c|c|c|c|c|c}
	2 & 4 & 1 & 5 & 6 & 9 & 8 & 11 & 7 & 12 & 3 & 13 & 10 \\
	\hline
	14.34 & 13.36 & 12.78 & 12.01 & 8.72 & 6.24 & 5.80 & 5.52 & 5.19 & 4.97 & 4.44 & 4.20 & 2.43 \\
\end{tabular}
\caption{The resulting ranking from requirements analysis by AHP}
\label{ahp}
\end{table}
\FloatBarrier

\subsection{Functional Requirements}
\begin{enumerate}
\item Critical
The categorized functional requirements for our web application are summarized below. Our critical requirements defines the functionalities needed to have a basic working product. The critical requirements speak to the purpose of our project, which is providing an accessible tool for individuals to create projects, collaborate with team members, and manage their tasks.
	\begin{enumerate}
	\item The system will allow users to create teams and assign members to those teams as well as add a project lead who will oversees the project. 
	\item The system will allow users to assign tasks to team members.
	\item The system will allow users to define project goals and requirements.
	\item The system will have a logical flow between tasks based on dependencies between tasks, calendars, and time progression. 
	\item The system will provide a way for users to view all tasks related to the project.
	\item The system will allow teams to create milestones within their project timeline.
	\end{enumerate}
\item Recommended
The recommended requirements describe additional functionality that we would would want to have in our project. Our recommended requirements outline animations and personal integrations that would enhance our project- but are not needed for a basic working system. 
	\begin{enumerate}
	\item The system will have a progress bar that will dynamically change as the project progresses.
	\item The system will have a personal calendar integration, so members not only can see the availability of other members, but also plan their schedules accordingly.
	\item The system will allow the user to create his/her own private set of tasks, schedules, etc.
	\end{enumerate}
\item Suggested
The suggested requirements explain the functionality that we would like to have but are last priority in our project. 
	\begin{enumerate}
	\item The system will provide users with a Gantt-style chart outlining milestones and current progress towards those milestones.
	\item The system will allow the project lead to have managerial privileges such as approving completion of tasks
	\item The system will allow users to traverse between milestones to show progress at those times.
	\item The system will teach basic project management skills.
	\end{enumerate}
\end{enumerate}
\subsection{Non-functional Requirements}
In addition to functional requirements, we have outlined nonfunctional requirements, summarized below. Because we want to appeal to a variety of audiences, we designed our application to be effortless and user-friendly. We want our users to engage in a collaborative and interactive environment. Because we want our application to be accessible, we designed our project to work on major web browsers as well as be compatible with desktops, laptops and tablets.
\begin{enumerate}
\item The system will be user-friendly.
\item The system will be aesthetically pleasing.
\item The system will be easy to use.
\item The system will be collaborative.
\item The system will work on major web browsers. 
\item The system will be compatible with desktops, laptops, and tablets. 
\end{enumerate}
\subsection{Design Constraints}
With regards to design constraints, we have identified one constraint. Our application must be a web-based application because of senior design requirements for the web design and engineering major. However, we believe that a web-based system will allow our application to be easily accessible for our users. 
\begin{enumerate}
\item The system must be a web-based system.
\end{enumerate}
