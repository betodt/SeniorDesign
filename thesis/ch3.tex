\chapter{Project Management}

\section{Test Plan}
We have divided our test plan into two categories: white box and black box testing. We plan to constantly conduct white box testing throughout project development. Because we have knowledge of our code, we will verify our system through unit testing to ensure that components are behaving correctly. Uniting testing will involve the login and logout process and project creation process. We will test the application\textsc{\char13}s logical flow of tasks based on dependencies between tasks, calendars, and time progression by analyzing our logical path models as well as performing unit testing. In addition, we will conduct black box testing by playing the role of a user and interacting with the system. Furthermore, we will engage in weekly inspections and reviews to examine the software artifact for the purpose of finding errors. \par To test usability, functionality and aesthetics, we will conduct acceptance testing through a series of alpha and beta testing. For alpha testing, we plan to observe users testing the account and project creation process. We will have the list of activities that we will ask the user to perform and then record the user\textsc{\char13}s feedback. In addition, we will conduct beta testing by asking users to use our application for managing their own small projects. After they complete their projects, they will answer survey questions in regards to functionality and user experience.

\section{Project Risks}
Risks threaten the success of our project. Issues are prone to happen during project development, thus we must be prepared when facing adversity. By conducting a risk analysis, we will can identify all possible risks and the impact on our application as a whole. As shown in table \ref{riskAnal}, we have listed each risk and its consequence. Next, we rated the probability from a scale of 0 to 1. In addition, we rated the severity of the risk from a scale of 0 to 10. We then multiplied the probability and severity to determine the risk\textsc{\char13}s impact to our project. After calculating the impact of each risk, we provided two mitigation strategies: one strategy to reduce the probability and another strategy to reduce the severity of the risk.
\FloatBarrier
\begin{table}%
\caption{Risk analysis table}\label{riskAnal}
\begin{tabularx}{\textwidth}{|>{\hsize=.5\hsize}X|X|c|c|c|>{\hsize=1.5\hsize}X|}
    \hline
    \textbf{Risk} & \textbf{Consequence} & \textbf{Probability} & \textbf{Severity} & \textbf{Impact} & \textbf{Mitigation Strategies}\\
    \hline
    Bugs & Delays in development timeline, resulting in failure to meet critical deadlines & 1 & 7 & 7 & Conduct peer code review sessions and use best coding practices to efficiently read and debug code.\par Ensure our code has low coupling. \\
    \hline
    Missed deadlines & Delay project, increasing the risk for project failure. & 0.4 & 10 & 4 & Set team deadlines in advance of actual deadlines in order to have a buffer period. Build the basic foundation for a working system and later conduct multiple releases to add functionality. \\
    \hline
    Data loss & Must rebuild code resulting in wasted time & 0.01 & 10 & 1 & Backup project to GitHub. Have multiple backups in different locations such as our own personal devices and on GitHub. \\
    \hline
    Team Dynamics & Fracture cohesiveness, resulting in poor teamwork. Wasted time to settle dispute. & 0.3 & 5 & 1.5 & Open communication with all group members. Conduct weekly meetings to ensure all members are cohesive.Conduct Gantt chart and assign responsibilities in beginning stages of the project, so if there is a dispute, each member is still accountable for his/her deliverables. \\
    \hline
\end{tabularx}
\end{table}%
\begin{table}%
\caption{Risk analysis table (cont'd.)}\label{riskAnalCont}
\begin{tabularx}{\textwidth}{|>{\hsize=.5\hsize}X|X|c|c|c|>{\hsize=1.5\hsize}X|}
    \hline
    \textbf{Risk} & \textbf{Consequence} & \textbf{Probability} & \textbf{Severity} & \textbf{Impact} & \textbf{Mitigation Strategies}\\
    \hline  
    Technology Issues & The technology used in the project do not provide the functionality that we hoped it would do. & 0.5 & 7 & 3.5 & Conduct extensive research about each technology we will use in our project to guarantee that the technology is right for our project purposes. Research other technologies that may provide as an alternative to our chosen technology. \\
    \hline
    User Issues & User dissatisfied. Must redo aspects of our project. Can lead to delays to project timeline & 0.5 & 8 & 4 & Conduct user studies ensure that we are following the requirements and building the right system for the user. Conduct user test throughout our project development, thus we can fix user issues instantly rather than allowing issues to accumulate at the end. \\ 
    \hline
    Scope Creep & Too many features, resulting in project release delays. & 0.75 & 7 & 5.25 & Prioritize requirements in order to focus on developing the most critical features of the project. Build the foundation of our project to ensure that we always have a working product. By adding functionality in later releases, we will have a product that is able to be released at any given moment. \\
    \hline
\end{tabularx}
\end{table}
\FloatBarrier
\section{Development Timeline}
In order to manage our time efficiently, we developed a Gantt chart. A Gantt chart displays the amount of work done or production that is completed in certain periods of time in relation to the amount planned for in such periods. Organization is crucial to the success of this project, thus we assigned each member responsibilities and set concrete deadlines for each responsibility. The following figures show our proposed development timeline.

\begin{figure}[h]
\centering
    \begin{gantt}{4}{2}
        \begin{ganttitle}
        \titleelement{Legend}{2}
        \end{ganttitle}
        \ganttbar[color=cyan]{Amy Tran}{0}{2}
        \ganttbar[color=red]{Alberto Diaz}{0}{2}
        \ganttbar{All}{0}{2}
    \end{gantt}
\end{figure}

\begin{figure}[h]
\centering
  \begin{gantt}{16}{11}
    \begin{ganttitle}
    \titleelement{Fall Quarter (Weeks 1 - 10)}{11}
    \end{ganttitle}
    \begin{ganttitle}
    \numtitle{1}{1}{11}{1}
    \end{ganttitle}
    \ganttgroup{Design Document}{0}{11}
    	\ganttbar{Problem Statement}{0}{3}
    	\ganttbar{Activity Diagram}{3}{1}
    	\ganttbar{Architectural Diagram}{3}{1}
		\ganttbar[color=red]{Use Case Diagrams}{3}{1}
		\ganttbar{Mock Ups}{4}{1}

		\ganttbar[color=cyan]{Requirements}{5}{1}
		\ganttbar[color=cyan]{Conceptual Model}{5}{1}
		\ganttbar[color=red]{Technologies Used}{6}{1}
		\ganttbar[color=red]{Design Rationale}{6}{1}
		\ganttbar[color=cyan]{Test Plan}{7}{1}
		\ganttbar[color=cyan]{Risk Analysis}{8}{1}
		\ganttbar{Begin Project Development}{9}{1}
		\ganttbar{Design Document Review}{10}{1}
  \end{gantt}
  \caption{The Fall quarter development timeline}
\end{figure}

\begin{figure}[h]
\centering
  \begin{gantt}{12}{10}
    \begin{ganttitle}
    \titleelement{Winter Quarter (Weeks 1 - 10)}{10}
    \end{ganttitle}
    \begin{ganttitle}
    \numtitle{1}{1}{10}{1}
    \end{ganttitle}
    \ganttgroup{Design Review}{0}{2}
    \ganttbar{Revised Design Report}{0}{3}
    \ganttgroup{Operational System}{3}{7}
        \ganttbar[color=red]{Login Portal}{3}{1}
        \ganttbar[color=cyan]{Project Creation Process}{4}{1}
        \ganttbar[color=red]{Team Creation Process}{5}{1}
        \ganttbar[color=cyan]{Tasks Assignment Process}{6}{2}
        \ganttbar[color=red]{Progress Bar and Milestones}{8}{1}
        \ganttbar{Front-End and Design}{3}{7}
        \ganttbar{Unit Testing}{3}{7}
  \end{gantt}
  \caption{The Winter quarter development timeline}
\end{figure}

\begin{figure}[h]
\centering
  \begin{gantt}{17}{10}
    \begin{ganttitle}
    \titleelement{Spring (Weeks 1 - 10)}{10}
    \end{ganttitle}
    \begin{ganttitle}
    \numtitle{1}{1}{10}{1}
    \end{ganttitle}

    \ganttgroup{Senior Thesis}{0}{10}
        \ganttbar[color=cyan]{User Manual}{1}{1}
        \ganttbar[color=red]{API Documentation}{2}{1}
        \ganttbar[color=red]{Maintenance Guide}{3}{1}
        \ganttbar[color=cyan]{Suggested Changes}{8}{1}
        \ganttbar[color=cyan]{Lessons Learned}{5}{1}
    \ganttgroup{Operational System}{0}{9}
       \ganttbar{Front-End and Design}{0}{9}
       \ganttbar[color=red]{Calendar Integration}{1}{1}
    \ganttgroup{Testing}{2}{4}
        \ganttbar{Acceptance Testing}{2}{1}
        \ganttbar{Testing Results}{3}{1}
        \ganttbar{Final Changes}{4}{2}
    \ganttgroup{Design Conference}{7}{1}
    \ganttbar{Final Changes}{8}{1}
    \end{gantt}
    \caption{The Spring quarter development timeline}
\end{figure}
\FloatBarrier