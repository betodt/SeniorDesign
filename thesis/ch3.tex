\chapter{Project Management}

\section{Test Plan}
We have divided our test plan into two categories: white box and black box testing. We plan to constantly conduct white box testing throughout project development. Because we have knowledge of our code, we will verify our system through unit testing to ensure that components are behaving correctly. Uniting testing will involve the login and logout process and project creation process. We will test the application\textsc{\char13}s logical flow of tasks based on dependencies between tasks, calendars, and time progression by analyzing our logical path models as well as performing unit testing. In addition, we will conduct black box testing by playing the role of a user and interacting with the system. Furthermore, we will engage in weekly inspections and reviews to examine the software artifact for the purpose of finding errors. \par To test usability, functionality and aesthetics, we will conduct acceptance testing through a series of alpha and beta testing. For alpha testing, we plan to observe users testing the account and project creation process. We will have the list of activities that we will ask the user to perform and then record the user\textsc{\char13}s feedback. In addition, we will conduct beta testing by asking users to use our application for managing their own small projects. After they complete their projects, they will answer survey questions in regards to functionality and user experience.

\section{Project Risks}
Risks threaten the success of our project. Issues are prone to happen during project development, thus we must be prepared when facing adversity. By conducting a risk analysis, we will can identify all possible risks and the impact on our application as a whole. As shown in table \ref{riskAnal}, we have listed each risk and its consequence. Next, we rated the probability from a scale of zero to one. In addition, we rated the severity of the risk from a scale of zero to ten. We then multiplied the probability and severity to determine the risk\textsc{\char13}s impact to our project. After calculating the impact of each risk, we provided two mitigation strategies: one strategy to reduce the probability and another strategy to reduce the severity of the risk.
\FloatBarrier
\begin{table}%
\caption{Risk analysis table}\label{riskAnal}
\begin{tabularx}{\textwidth}{|>{\hsize=.5\hsize}X|X|c|c|c|>{\hsize=1.5\hsize}X|}
    \hline
    \textbf{Risk} & \textbf{Consequence} & \textbf{Probability} & \textbf{Severity} & \textbf{Impact} & \textbf{Mitigation Strategies}\\
    \hline
    Bugs & Delays in development timeline, resulting in failure to meet critical deadlines & 1 & 7 & 7 & Conduct peer code review sessions and use best coding practices to efficiently read and debug code.\par Ensure our code has low coupling. \\
    \hline
    Missed deadlines & Delay project, increasing the risk for project failure. & 0.4 & 10 & 4 & Set team deadlines in advance of actual deadlines in order to have a buffer period. Build the basic foundation for a working system and later conduct multiple releases to add functionality. \\
    \hline
    Data loss & Must rebuild code resulting in wasted time & 0.01 & 10 & 1 & Backup project to GitHub. Have multiple backups in different locations such as our own personal devices and on GitHub. \\
    \hline
    Team Dynamics & Fracture cohesiveness, resulting in poor teamwork. Wasted time to settle dispute. & 0.3 & 5 & 1.5 & Open communication with all group members. Conduct weekly meetings to ensure all members are cohesive.Conduct Gantt chart and assign responsibilities in beginning stages of the project, so if there is a dispute, each member is still accountable for his/her deliverables. \\
    \hline
\end{tabularx}
\end{table}%
\begin{table}%
\caption{Risk analysis table (cont'd.)}\label{riskAnalCont}
\begin{tabularx}{\textwidth}{|>{\hsize=.5\hsize}X|X|c|c|c|>{\hsize=1.5\hsize}X|}
    \hline
    \textbf{Risk} & \textbf{Consequence} & \textbf{Probability} & \textbf{Severity} & \textbf{Impact} & \textbf{Mitigation Strategies}\\
    \hline  
    Technology Issues & The technology used in the project do not provide the functionality that we hoped it would do. & 0.5 & 7 & 3.5 & Conduct extensive research about each technology we will use in our project to guarantee that the technology is right for our project purposes. Research other technologies that may provide as an alternative to our chosen technology. \\
    \hline
    User Issues & User dissatisfied. Must redo aspects of our project. Can lead to delays to project timeline & 0.5 & 8 & 4 & Conduct user studies ensure that we are following the requirements and building the right system for the user. Conduct user test throughout our project development, thus we can fix user issues instantly rather than allowing issues to accumulate at the end. \\ 
    \hline
    Scope Creep & Too many features, resulting in project release delays. & 0.75 & 7 & 5.25 & Prioritize requirements in order to focus on developing the most critical features of the project. Build the foundation of our project to ensure that we always have a working product. By adding functionality in later releases, we will have a product that is able to be released at any given moment. \\
    \hline
\end{tabularx}
\end{table}
\FloatBarrier
\section{Development Timeline}
In order to manage our time efficiently, we developed a Gantt chart. A Gantt chart displays the amount of work done or production that is completed in certain periods of time in relation to the amount planned for in such periods. Organization is crucial to the success of this project, thus we assigned each member responsibilities and set concrete deadlines for each responsibility. The following figures show our proposed development timeline.

\begin{figure}[h]
\centering
    \begin{gantt}{4}{2}
        \begin{ganttitle}
        \titleelement{Legend}{2}
        \end{ganttitle}
        \ganttbar[color=cyan]{Amy Tran}{0}{2}
        \ganttbar[color=red]{Alberto Diaz}{0}{2}
        \ganttbar{All}{0}{2}
    \end{gantt}
\end{figure}

\begin{figure}[h]
\centering
  \begin{gantt}{16}{11}
    \begin{ganttitle}
    \titleelement{Fall Quarter (Weeks 1 - 10)}{11}
    \end{ganttitle}
    \begin{ganttitle}
    \numtitle{1}{1}{11}{1}
    \end{ganttitle}
    \ganttgroup{Design Document}{0}{11}
    	\ganttbar{Problem Statement}{0}{3}
    	\ganttbar{Activity Diagram}{3}{1}
    	\ganttbar{Architectural Diagram}{3}{1}
		\ganttbar[color=red]{Use Case Diagrams}{3}{1}
		\ganttbar{Mock Ups}{4}{1}

		\ganttbar[color=cyan]{Requirements}{5}{1}
		\ganttbar[color=cyan]{Conceptual Model}{5}{1}
		\ganttbar[color=red]{Technologies Used}{6}{1}
		\ganttbar[color=red]{Design Rationale}{6}{1}
		\ganttbar[color=cyan]{Test Plan}{7}{1}
		\ganttbar[color=cyan]{Risk Analysis}{8}{1}
		\ganttbar{Begin Project Development}{9}{1}
		\ganttbar{Design Document Review}{10}{1}
  \end{gantt}
  \caption{The Fall quarter development timeline}
\end{figure}

\begin{figure}[h]
\centering
  \begin{gantt}{12}{10}
    \begin{ganttitle}
    \titleelement{Winter Quarter (Weeks 1 - 10)}{10}
    \end{ganttitle}
    \begin{ganttitle}
    \numtitle{1}{1}{10}{1}
    \end{ganttitle}
    \ganttgroup{Design Review}{0}{2}
    \ganttbar{Revised Design Report}{0}{3}
    \ganttgroup{Operational System}{3}{7}
        \ganttbar[color=red]{Login Portal}{3}{1}
        \ganttbar[color=cyan]{Project Creation Process}{4}{1}
        \ganttbar[color=red]{Team Creation Process}{5}{1}
        \ganttbar[color=cyan]{Tasks Assignment Process}{6}{2}
        \ganttbar[color=red]{Progress Bar and Milestones}{8}{1}
        \ganttbar{Front-End and Design}{3}{7}
        \ganttbar{Unit Testing}{3}{7}
  \end{gantt}
  \caption{The Winter quarter development timeline}
\end{figure}

\begin{figure}[h]
\centering
  \begin{gantt}{17}{10}
    \begin{ganttitle}
    \titleelement{Spring (Weeks 1 - 10)}{10}
    \end{ganttitle}
    \begin{ganttitle}
    \numtitle{1}{1}{10}{1}
    \end{ganttitle}

    \ganttgroup{Senior Thesis}{0}{10}
        \ganttbar[color=cyan]{User Manual}{1}{1}
        \ganttbar[color=red]{API Documentation}{2}{1}
        \ganttbar[color=red]{Maintenance Guide}{3}{1}
        \ganttbar[color=cyan]{Suggested Changes}{8}{1}
        \ganttbar[color=cyan]{Lessons Learned}{5}{1}
    \ganttgroup{Operational System}{0}{9}
       \ganttbar{Front-End and Design}{0}{9}
       \ganttbar[color=red]{Calendar Integration}{1}{1}
    \ganttgroup{Testing}{2}{4}
        \ganttbar{Acceptance Testing}{2}{1}
        \ganttbar{Testing Results}{3}{1}
        \ganttbar{Final Changes}{4}{2}
    \ganttgroup{Design Conference}{7}{1}
    \ganttbar{Final Changes}{8}{1}
    \end{gantt}
    \caption{The Spring quarter development timeline}
\end{figure}
\FloatBarrier

\section{Societal Issues}
When building a new application, it is important to realize the societal aspects pertaining to the new product. As engineers, we have the ability to influence our users and thus we must act in an ethical manner when designing and developing an application. In order to ensure that we were acting ethically, we referenced the ACM Code of Ethics throughout the process of our application. Below is a ethical justification for our project, as well as how we plan to ensure that we are acting ethically with our team members and during product development. Lastly, we discuss social and cultural issues associatied with Planly. 
\subsection{Ethical Justification for Our Project}
 We selected to build a collabortive project management application because we noticed a need for a management and organizational application for individuals who have multiple commitments in their lives. The moral reason we decided on our project is because we are aware that we live in a society based on productivity. Many individuals are being bombarded with tasks from different organizations, and with the invention of mobile devices, it seems that work never stops. We believe all humans have the right to live a dignified life and should not be stressed to such severity that health-related consequences, such as mental illness, can occur. With the creation of an organizational management application, we hope to relieve stress by providing individuals with an accessible web based interface where they can easily communicate with their team members, find an organized list of their tasks and keep track of their schedules. In addition, we hope that individuals will use our project management app to plan personal projects, such as weddings or birthday parties, thus allowing them to enjoy their lives doing the things that make life memorable. 
 \subsection{Team and Organizational Ethics}
 Our team is composed of two individuals. Practical steps that we have taken to ensure fair treatment among team members is by effectively communicating and negotiating which tasks each member will do. We ensure that we are acting ethically by obeying the IEEE-CS/ACM Software Engineering Code of Ethics (later explained in the Product Development section) and also by using the SCU’s Ethical Decision Framework when we are unsure of a decision. 
 \subsection{Product Development} 
To ensure that we are acting ethically when developing our product, we followed the  IEEE-CS/ACM Software Engineering Code of Ethics. 

 \begin{center}
 \begin{tabular}{|c c|} 
 \hline
  IEEE-CS/ACM Code Of Ethics & Our Team  \\ [0.5ex] 
 \hline\hline
 Public: Act in the public's interest & When deciding our project, we analyzed who our market is and what their needs are to ensure we acting in the public’s interest. \\ 
 \hline
 2 & 7 & 78 & 5415 \\
 \hline
 3 & 545 & 778 & 7507 \\
 \hline
 4 & 545 & 18744 & 7560 \\
 \hline
 5 & 88 & 788 & 6344 \\ [1ex] 
 \hline
\end{tabular}
\end{center}

\subsection{Social and Cultural Issues}
\par We have an ethical duty to do no harm to the potential users of our product. When building our system, we will take extremely proactive measures to ensure that their data is safe and secure. We would never release any personal data unless there is legal action in where we will be sure to inform our user before doing so. When conducting product test, we will always provide our testers with a consent form and ensure that they know what they are signing off to. We would never pressure anyone to use our product. A ramification of our product to society as a whole is that while it will help individuals stay organization, it may cause individuals to become too dependent on technology. While we want people to use our product to ease their lives, we hope that they see our tool as an aid rather than a dependency.
\subsection{Political Issues}
\par We do not foresee any political issues with our system because we do not see any need for political representatives to become involved with our application.
\subsection{Economic Issues}
\par We do not foresee any economic issues with our system as is it not making any revenue. 
\subsection{Health and Safety Issues}
\par A potential health issue that we see with our system is creating a dependency on technology; however, as we stated in the Social and Cultural Issues section, we hope that individuals use our system as an aid rather than a dependency. As for safety issues, one issue is if our system were to be attacked and our users' data was captured, this can cause potential harm to our users' saftey since their personal information is now exposed. 
\subsection{Manufacturability Issues}
\par We do not foresee any manufacturability issues because our product is not manufacturable.
\subsection{Sustainability Issues}
\par Because our product is a software product, we do not foresee any sustainability issues.
\subsection{Environmental Issues}
\par As stated in the Sustainability Issue, our product is a software product, thus we do not believe that our system will cause environmental issues.
\subsection{Usability Issues}
\par Because our application is very client heavy, we knew that usability would be the key to our application's success. Throughout the developement process, we were sure to include the user in every step and iterated until the user was satisfied. We were able to recieve feedback and examine user interactions with each component of our system. In addition, at the end of development, we hosted a user study, discussed in the conclusion section. 
\subsection{Lifelong Learning}
Through developing our system, we engaged in lifelong learning. Knowledge from our classes was not enough for us to succeed. The project inspired us to study and learn new technologies, such as Ember.js and Firebase, as well as strengthen our existing skills. By completing this project, we feel prepared for the time where we must learn on our own. 